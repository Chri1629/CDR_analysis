\documentclass[10pt, a4paper,openany]{article}
\usepackage[italian]{babel}
\usepackage[T1]{fontenc}
\usepackage[table]{xcolor}
\usepackage{float}
\restylefloat{table,figure}
\usepackage{graphicx}	
\usepackage[utf8]{inputenc}
\usepackage{amsmath}
\usepackage{fancyhdr}
\usepackage{geometry}
\geometry{a4paper,top=2cm,bottom=2cm,left=3cm,right=3cm,%
	heightrounded,bindingoffset=5mm}

\usepackage{amssymb}
\usepackage{amsthm}
\usepackage{multicol}
\usepackage{xcolor}

\definecolor{ultramarine}{RGB}{	255,0,0} 

\begin{document}
\begin{center}
\huge\textbf{\textcolor{ultramarine}{Si può stimare la predisposizione di una persona all'alzheimer?}}

Un'analisi delle caratteristiche utili per predire la predisposizione di una persona all'alzheimer.
\end{center}

\begin{center}
Vittorio Bomba, Federico Luzzi,  Marco Peracchi, Christian Uccheddu
\end{center}

\hrule
\vspace{0.3cm}

\begin{center}\textbf{{Introduzione e obiettivi}}
\\
In questo lavoro l'obiettivo è quello di capire se si possano predire malattie quali la demenza senile e l'alzheimer a partire da una serie di caratteristiche misurabili in precedenza. Per raggiungere questo scopo abbiamo usato tecniche di classificazione per cercare di classificare un individuo come malato di demenza senile o sano, abbiamo inoltre usato delle tecniche di regressione per cercare di stimare l'indice corrispondente al grado di demenza.
\vspace{0.3cm}
\hrule
\end{center}
\begin{multicols}{2}
	
\section*{Introduzione}
\section*{Descrizione Dataset}
\section*{Strutturazione Dataset}
\section*{Clustering}
\section*{Validazione}
\section*{Interpretazione dei risultati}
\begin{table}
	\caption{h. . .i}
	\label{tab:esempio}
	\centering
	\begin{tabular}{H}
		...
	\end{tabular}
\end{table}
\section*{Conclusioni}

\end{multicols}

\end{document}